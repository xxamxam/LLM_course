\ \\\textbf{1. MSE}
\\MSE - mean squared error. Среднеквадратичное отклонение прогноза от исходного значения. Сильнее штрафует за
бОльшие по модулю отклонения.
$$MSE =\frac{1}{n}\sum\limits_{i=1}^{n}(Y_i -\hat{Y_i})^2$$
где $\hat{Y}$ - предсказанный результат, Y - реальный.
\\
\\
\textbf{2. MAE}
\\
MAE - mean absolute error. Отклонение прогноза от исходного значение, усреднённое по всем наблюдениям.
$$MAE =\frac{1}{n}\sum\limits_{i=1}^{n}|Y_i -\hat{Y_i}|$$
Среднеквадратичный функционал сильнее штрафует за большие отклонения по сравнению со среднеабсолютным, и
поэтому более чувствителен к выбросам.
Среднеквадратичная ошибка подходит для сравнения двух моделей или для контроля качества во время обучения, но
не позволяет сделать выводов о том, насколько хорошо данная модель решает задачу. Например, MSE = 10 является очень
плохим показателем, если целевая переменная принимает значения от 0 до 1, и очень хорошим, если целевая переменная
лежит интервале (10000, 100000).
\\
\\
\textbf{3. $R^2$} 
$$R^2 = 1 -
\frac{\sum\limits_{
i=1}^{n}(Y_i - \hat{Y_i})^2}{\sum\limits_{
i=1}^{n}(Y_i - \overline{Y})^2}$$

Член вычитаемый из 1 можно интерпретировать как оценённую дисперсию, отнесённую к реальной дисперсии. Ещё можно сказать, что мы модель сравниваем с моделью, которая предсказывает просто константу. Смотрим насколько далеко ушла наша модель от тупого предсказания среднего.
Может иметь отрицательные значения, так как числитель может неограниченно расти.
\begin{unnecessary}
\ \\
\textbf{MAPE}\\
$$MAPE =\frac{1}{n}\sum\limits_{
i=1}^{n}|\frac{Y_i - \hat{Y_i}}{Y_i}|$$
\\
\textbf{SMAPE}
\\
 Когда есть большие выбросы в MAPE, следует
использовать отнормированную и на прогнозируемое значение метрику.
$$SMAPE =\frac{1}{n}\sum\limits_{
i=1}^{n}\frac{2|Y_i - \hat{Y_i}|}{|Y_i| + |\hat{Y_i}|}$$
\end{unnecessary}
