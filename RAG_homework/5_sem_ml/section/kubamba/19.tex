\textbf{Random Subspace Method (RSM):} случайно выбираем некоторые наборы признаков, обучаем модель только на них и усредняем результаты. Из-за того, что наборы признаков разные, ошибки не очень похожи друг на друга, а значит мы приближаемся к теоретическому результату из прошлого билета

\textbf{Random Forest:} объединим bagging и RSM на деревьях, чтобы ошибки стали еще менее похожими

Плюсы random forest: нелинейная разделяющая поверхность, сложно переобучить (переобучения всех деревьев усредняются, проблемы могут быть только если возьмем огромное количество деревьев и наборы признаков будут повторяться), не ломается при наличии скоррелированных признаков, деревья хорошо работают с пропусками (если нет признака, который нужен для этого листа, просто считаем значения в каждой из веток и усредняем)

\[ \hat{y} = \frac{|L|}{|Q|} \hat{y_L} + \frac{|R|}{|Q|} \hat{y_R}\]

Так же random forest позволяет использовать обучающую выборку для валидации: так как мы обучаем каждое дерево на бутстрапной выборке, есть объекты, которые каждая из моделей не видела. Поэтому можно для каждого объекта взять только те модели, которые его не видели и усреднить предсказания по ним. Получим оценку сверху на ошибку (так как уменьшили ансамбль, следовательно ошибка выросла).

\[ OOB = \sum_{i=1}^m L\left(y_i, \frac 1 {\sum_{n=1}^N I[x_i \not\in X_n]} \sum_{n=1}^N I[x_i \not\in X_n] b_n(x_i)\right) \quad \text{(out-of-bag error)} \]

\begin{lemmanote}
    Если данные упорядочены во времени надо быть очень осторожными с OOB, потому что важно смотреть распределение <<новое при условии старого>>, а не наоборот (иначе модель обучается чему-то не тому)
\end{lemmanote}
\textbf{Модификации:}
\begin{enumerate}
    \item Extremely Randomized Trees: выбираем разделения в листах более рандомно (в экстремальном случае: деревья даже не зависят от ответов обучающей выборки)

    \item Isolation forest (метод поиска аномалий): не предсказываем что-то, а пытаемся отделить аномалии, так как их проще отделить деревом чем остальные точки (этот метод не универсальный, работает не на всех задачах. Пример: точки на плоскости, <<свернули в рулон>> и добавили аномалии между слоями. Метод будет работать плохо)
\end{enumerate}